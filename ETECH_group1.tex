\documentclass[12pt, oneside]{article}   	
\usepackage{geometry}
\geometry{a4paper}
\usepackage[onehalfspacing]{setspace}

\usepackage{graphicx}
\usepackage[labelfont=bf,font=small,singlelinecheck=false]{caption}
\usepackage{prettyref}
\newrefformat{fig}{Figure~\ref{#1}}

%horizontal line
\newcommand{\HRule}{\rule{\linewidth}{0.25mm}}

\begin{document}

%TITLE PAGE
\begin{titlepage}
\begin{center}

\Large{\textsc{Feasibility study}}

\HRule \\[0.35cm]
{\Huge \bfseries The Monitoring of Crowd Movement\\}
\HRule 

\small (working title)\\[0.5cm]

\large Omar Amjad, Josie Hughes, Philip Mair, \\ James Manton, Tiesheng Wang\\[1.0cm]
January 2015\\[1.0cm]

Progress Report\\
for the MRes ETECH Project\\

\vfill

\end{center}
\end{titlepage}

\section{Discovery}

We identified three main fields the different stakeholders will be interested in. First, there is the obvious commercial interest. Second, there is an interested in improved customer experience. Third, there may be concerns of data safety and protection. The main stakeholders are grouped according to these fields in \prettyref{fig:fields}, a description of each partaker now follows.

\paragraph{We:} Our company needs to make profit. It will needs to attract customers by offering them value in the form of improved experience for their own customers.

\paragraph{Customer:} The customer will want our product if they can improve the services they offer to make more money. They may be medium to large corporations so they can afford a (currently) non-essential crowd monitoring system. The technology itself is of no interest to them, they are rather interested in the analysed well presented data we gather and offer. 

\paragraph{Crowd:} The crowd we monitor is not directly interacting with us, but is central to the succes of our business. We can convince them of the utility of our products if we their experience is enhanced while we minimally interfere with their general habits. They may want to track their own interactions at, for example, at conferences. A concern the crowd may have is the protection and safety of their personal data. This concern can be preemted by anonymising the data at the point of recording and/or making non-anonymised data only available to the person it belongs to. There is a strong need for transparency about this process to reassure the crowd. Data ownership questions have to be clarified.

\paragraph{Government:} We will need to comply with data protection law and record evidence for doing so, for example on how the data is stored and transferred. There is a potential partnership: In its efforts to improve emergency procedures and policy the government may be interested in our expertise of understanding people's behaviour in large assemblies under stress conditions.

\paragraph{Supplier:} Our suppliers will need us to pay our bills. Again, there is potential for partnerships. For example, a well known supplier partnering with us could increase our credibility building trust among customers. 

\paragraph{Activists:} We may attract the attention of data protection activists.

\section{Belonging} 
The section can be summarised into the following three sections. 
 
\paragraph{What can be offered?} The movement monitoring system can provide real-time information about flow of people. For our customer, the information can be used to effectively monitor people density so as to prevent potential problems caused by congestion. The crowd can use the information about people density to make a decision for themselves about the place to go. It is believed that this self-monitoring approach can effectively prevent overcrowding. The system can also enhance the safety of the monitored area by identifying the dangerous people. Meanwhile, by studying trends and patterns of information, our customer can optimise their services.

\paragraph{What are potential issues?} Big-brother situation is likely to be raised by the monitoring system. Our customer may use the information to control the crowd which may disobey the crowd's willing. For instance, the system may stimulate the dissent between employees and employers. There are some privacy infringements as well. The ambivalence about utilising the monitoring system may be raised from the privacy issues. Some people may feel reluctant towards the deployment of the system. Another potential issue may come from data protection. Any unexpected information leakage is likely to have significant negative impact on public and our company.

\paragraph{What shall we do?} Our system should let both the crowd and our costumer feel sufficiently convenient and comfortable. We need to discuss the transparency of both information acquisition (i.e. Should the crowd know they are monitored?) and information handling (i.e. Should the crowd know who use the information for what?) We also need to discuss the level of detail for monitoring to mitigate any privacy infringement. If the crowd should know about the presence of monitoring system, we can offer them a choice for being monitored or not to alleviate potential issues mentioned above.  We need to consider the conditions of participation to balance the interest of each stakeholder. e.g. The crowd may able to access to the monitoring information about themselves after participation. For the information repository, we need to build trust with our costumer and the crowd.

\begin{figure}
\centering
\includegraphics[width=0.5\textwidth]{./discovery-fields}
{\captionof{figure}[Text for figure list.]{\label{fig:fields} 
The interests of each major stakeholder and how they relate to one another are shown.}}
\end{figure}

\end{document}
